\documentclass{beamer}

\usetheme{Boadilla}
\usecolortheme{default}
\usepackage{amsmath}
\usepackage{tcolorbox}

\title{Aritmética del Punto Flotante}
\subtitle{IN 301 Clase 2}
\author{Uri Groisman}
\date{\today}

\begin{document}

% Slide 1
\begin{frame}
    \titlepage
\end{frame}

% Slide 2
\begin{frame}{Aritmética del Punto Flotante}
    \begin{enumerate}
        \item \textbf{Análisis numérico}
        \begin{itemize}
            \item definición
            \item fuentes de error
        \end{itemize}
        \item \textbf{Números en Punto flotante}
        \begin{itemize}
            \item representación en punto flotante de un número real 
            \item precisión de la máquina
            \item Jupyter notebook en Julia
        \end{itemize}
        \item \textbf{Aritmética del Punto Flotante}
        \begin{itemize}
            \item sumando dos números en Punto Flotante
            \item pérdida de significado
        \end{itemize}
        \item \textbf{Precisión Arbitraria y Aritmética de Intervalo}
        \begin{itemize}
            \item generalización de la aritmética de punto flotante
        \end{itemize}
    \end{enumerate}
\end{frame}

% Slide 3
\begin{frame}{Aritmética del Punto Flotante}
    \begin{enumerate}
        \item<1-> \textbf{Análisis numérico}
        \begin{itemize}
            \item<1-> definición
            \item<2-> fuentes de error
        \end{itemize}
        \item<3-> \textbf{Números en Punto flotante}
        \begin{itemize}
            \item representación en punto flotante de un número real 
            \item precisión de la máquina
            \item Jupyter notebook en Julia
        \end{itemize}
        \item<4-> \textbf{Aritmética del Punto Flotante}
        \begin{itemize}
            \item sumando dos números en Punto Flotante
            \item pérdida de significado
        \end{itemize}
        \item<5-> \textbf{Precisión Arbitraria y Aritmética de Intervalo}
        \begin{itemize}
            \item generalización de la aritmética de punto flotante
        \end{itemize}
    \end{enumerate}
\end{frame}


% Slide 4
\begin{frame}{Numerical Analysis – a definition}
    \begin{tcolorbox}[colback=blue!5!white,colframe=blue!75!black,title=Definition (Nick Trefethen, SIAM News 1992)]
        \textit{Numerical analysis} is the study of algorithms for the problems of continuous mathematics.
    \end{tcolorbox}
    \vspace{1em}
    An algorithm is a finite number of unambiguous steps, where each step can be executed by arithmetical operations.

    We care for the efficiency and accuracy of algorithms.

    In continuous models to solve problems, we obtain approximate answers for approximate input data.

    Two related disciplines:
    \begin{itemize}
        \item Computer Algebra to formulate and re-formulate problems.
        \item Scientific Computing, for applications to science.
    \end{itemize}
\end{frame}

% Slide 5
\begin{frame}{Floating-Point Arithmetic}
    \begin{enumerate}
        \item<1-> \textbf{Numerical Analysis}
        \begin{itemize}
            \item a definition
            \item sources of error
        \end{itemize}
        \item<2-> \textbf{Floating-Point Numbers}
        \begin{itemize}
            \item floating-point representation of a real number
            \item machine precision
            \item a Julia session in CoCalc
        \end{itemize}
        \item<3-> \textbf{Floating-Point Arithmetic}
        \begin{itemize}
            \item adding two floating-point numbers
            \item loss of significance
        \end{itemize}
        \item<4-> \textbf{Arbitrary Precision and Interval Arithmetic}
        \begin{itemize}
            \item extending floating-point arithmetic
        \end{itemize}
    \end{enumerate}
\end{frame}

% Slide 6
\begin{frame}{Sources of Error}
    Some sources of error are
    \begin{itemize}
        \item truncation errors in mathematical models;
        \item observed input data are approximate numbers;
        \item representation errors, e.g.: \(1/10\) in binary, \(1/3\) in decimal;
        \item roundoff error during calculations.
    \end{itemize}
    \vspace{1em}
    In numerical analysis, we ask two important questions:
    \begin{enumerate}
        \item How sensitive is the output to changes in the input?
        \item Do roundoff errors in an algorithm propagate?
    \end{enumerate}

    Answers to these two questions, are addressed respectively by
    \begin{itemize}
        \item numerical conditioning is a property of a problem;
        \item numerical stability is a property of an algorithm.
    \end{itemize}
\end{frame}

\end{document}
